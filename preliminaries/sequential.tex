A sequential protocol is a sequential program computed from a round-based protocol \Psync. It manipulates the following variables: an integer variable $n$, corresponding to the number of processes executing the protocol, for each variable $v$ of type $T$ in \Psync, it has $s_v$ an array of type $\mathit{ID} \rightarrow T$, where each index $i$ gives the value of the variable for process $p_i$.

The sequentialization \Pseq\ manipulates several arrays, each storing elements of some message type, and $\texttt{mbox}_{r}[p_i]$ is the value of $\texttt{mbox}$ in round $r$ on process $p_i$. 
The transition relation of \Pseq\ defines a total order over all actions performed by all processes, i.e., an order across all \texttt{send} and \texttt{update}.

Round-based protocols impose a total order over actions performed by processes across rounds. 
The sequentialization maintains this order, and it is mainly concerned with the code of one round. 
The sources of non-determinism at the round level are: 
(1) the order in which processes send messages 
(2) the order in which processes execute update 
(3) the order in which messages are received and
(4) which messages are received. 

The round-based semantics ensure that between any two processes $p$ and $q$ there is at most one message that is sent from $p$ to $q$ and vice versa.  
Consequently, the order in which send and receive actions are sequentialized does not matter. 

The update function takes the set of received messages as input, and performs a local computation. Two update functions of the same round, on different processes are independent.

Therefore, we fix one order across processes, denoted  $p_1, p_2, \ldots p_n$ where the index gives the order relation.
The calls to send and update are sequentialized according to this order, where all sends go before all updates. 

\begin{algorithm}[H]
\caption{Sequentialization}
\begin{algorithmic}[1]
\scriptsize

\For{$\mathtt{s=1\ to\ n}$} \Comment{RoundSeq}
    \For{$\mathtt{r=1\ to\ n}$}
        \State $\mathtt{mailbox_R(p_r)}$ += $\mathtt{(*)p_s.send(p_r)}$ \label{algorithm:sequentialization:choose}
    \EndFor
\EndFor

\For{$\mathtt{i=0\ to\ n}$}
    \State $\mathtt{p_i.update(mailbox_R(p_i))}$
    \State $\mathtt{mailbox_R(p_i)=\emptyset}$
\EndFor

\While{$\mathtt{true}$} \Comment{ProtocolSeq}
    \For{$\mathtt{R=1\ to\ K}$}
        \State $\mathtt{RoundSeq(R)}$
    \EndFor
\EndWhile
\end{algorithmic}
\label{algorithm:sequentialization}
\end{algorithm}

For each message sent the sequential program makes a non-deterministic choice whether to deliver it or not. 
Each \texttt{send-receive} pair is replaced with an assignment, that non-deterministically adds or not the sent message to the receiver's mailbox.

Algorithm~\ref{algorithm:sequentialization} uses ``$*$" to represent a non-deterministic choice, i.e., if the message sent by process $p_i$ to process $p_j$ is received by $p_j$. 

A protocol consisting of $K$ rounds is sequentialized in a while loop that executes the sequentialization of one round after another, in the order in which they are defined in the round-based protocol. 
