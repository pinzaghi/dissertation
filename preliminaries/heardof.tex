The syntax of round-based protocols \Psync\ consists of an initialization function \texttt{init} and a phase consisting of a non-empty finite sequence of rounds $r_1, ..., r_k$. 

All processes execute the initialization function followed by the given sequence of rounds in lock-step, in a loop. 

The round number is an abstract notion of time: all processes are in the same round. In each round, processes send messages in one synchronized step, using \texttt{SEND}. 

Each process receives in one atomic step a non-deterministically chosen subset of the messages that were sent to it. We denote by $\mathit{mailbox} : P \rightarrow 2^{\mathit{Msg}}$ the set of received messages in the current round per process.
Messages sent in a round, are either received in the same round or lost. 
All processes update the local state synchronously, using \texttt{UPDATE}.  

We introduce synchronization tags, a lightweight annotation for checking the existence of a round structure.



